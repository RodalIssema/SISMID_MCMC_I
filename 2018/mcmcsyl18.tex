\documentclass[11pt]{article}
\setlength{\oddsidemargin}{0cm}
\setlength{\textwidth}{16.5cm}
\setlength{\parskip}{0.2cm}
\setlength{\topmargin}{-1cm}
\setlength{\textheight}{22.8cm}
\begin{document}
\bibliographystyle{plain}
\addtolength{\baselineskip}{0.0\baselineskip}  
\pagestyle{myheadings} 
\markright{MCMC I:  \today} 

\begin{center}
{\bf  MCMC I \\
10th Summer Institute in Statistics and Modeling in Infectious Diseases \\
 Course Time Plan  \\
July 11-13, 2018}  
\end{center}

\vspace{.2cm}

\noindent
 {\bf Instructors:} Vladimir Minin,  Kari Auranen, M.\ Elizabeth Halloran\\

\vspace{.3cm}

\noindent
{\bf Course Description:} This module is an introduction to Markov chain Monte Carlo methods with some simple applications in infectious disease 
studies.  
 The course includes an introduction to 
 Bayesian inference, Monte Carlo, MCMC, some background theory, and convergence diagnostics. Algorithms include Gibbs sampling and Metropolis-Hastings and combinations. Programming is in R. 
 Familiarity with the R statistical package is assumed. 



\noindent
{\bf Course schedule:}
The course is composed of 10 90-minute sessions, for a total of 15 hours of instruction. 


\section{Introduction to Bayesian Inference} 
\begin{itemize} 
\item Overview of the course. 
\item Bayesian inference: Likelihood, prior, posterior, normalizing constant  
\item Conjugate priors; Beta-binomial; Poisson-gamma; normal-normal 
\item Posterior summaries, mean, mode,  posterior intervals   
\item Motivating examples: Chain binomial model (Reed-Frost), General Epidemic Model,  
SIS model.
\item {Lab: }
\begin{itemize}
\item Goals:  Warm-up with R for simple Bayesian computation
%item  Estimating a rate from Poisson data using a Gamma prior distribution (conjugate prior) 
\item  Example: Posterior distribution of transmission probability with a binomial  sampling distribution using a conjugate beta prior distribution
\item Summarizing posterior inference (mean, median, posterior quantiles and intervals)
\item Varying the amount of prior information
\end{itemize}
\end{itemize}

\section{Introduction to Gibbs Sampling}
\begin{itemize}
\item Chain binomial model and data augmentation
\item Brief introduction to Gibbs sampling
\item Lab
\begin{itemize}
\item Goals: Simple data augmentation using MCMC
\item Example: Gibbs sampler for the chain binomial model.
\end{itemize}
\end{itemize}


\section{Introduction to computation}
 \begin{itemize}
 \item Random number generators 
\item {Non-iterative Monte Carlo methods}
\begin{itemize}
\item Direct sampling, Monte Carlo integration (classical Monte Carlo)
\item Indirect methods: importance sampling, rejection sampling,
\end{itemize}
\item{Basic Markov Chain Theory} 
\begin{itemize}
\item Definitions
\item Stationarity
\item The ergodic theorem
\end{itemize}
\item Lab: 
\begin{itemize}
\item Goals: Importance sampling, Markov chain
\end{itemize} 
\end{itemize}

\section{Markov chain Monte Carlo methods}
\begin{itemize}
\item Gibbs sampling 
\begin{itemize}
\item Background 
\item Revisit simple Gibbs sampler for chain-binomial model 
\end{itemize}
\item{Metropolis-Hasting algorithm}
\item Lab: 
\begin{itemize}
\item Goals: M-H: elementary missing data imputation on S-I model
\end{itemize}
\end{itemize}

\section{Metropolis-Hasting and Gibbs combined} 
\begin{itemize}
\item Example: Hierarchical model
\item Lab:
\begin{itemize}
\item Goals: Combining Metropolis and Gibbs in one algorithm
\item Example: Beta-binomial hierarchical model with rat data
\end{itemize}
\end{itemize}

\section{Chain binomial model revisited} 
\begin{itemize} 
\item Hierarchical chain binomial model with hyperparameters
\begin{itemize}
\item  Model checking 
\item Allowing for heterogeneity
\end{itemize}
\item Lab: 
\begin{itemize}
\item Goals: Combined M-H and Gibbs and learning model checking
\item   Example: Hierarchical beta-binomial chain binomial model
\end{itemize}
\end{itemize}


\section{General Epidemic Model}
\begin{itemize}
\item The general epidemic model and incompletely observed data 
\item Algorithm
\item {Lab: General epidemic model}
\begin{itemize}
\item  Goals: parameter estimation with data augmentation
\item Example: smallpox transmission
\end{itemize}
\end{itemize}

\section{Diagnostics, etc}
\begin{itemize}
\item Assessing convergence (more or less), Coda 
\item Variance reduction, Monte Carlo error
\item Poisson process 
\item{Lab: Diagnostics}
\begin{itemize}
\item Goals: learn how to do basic diagnostics on chain and output
\item Coda 
\item Diagnostics on previous examples
\end{itemize}
\end{itemize}

\section{SIS model}
\begin{itemize}
\item  Binary Markov process model for a recurrent infection
\item Likelihood
%\item  Reversible jump MCMC 
\item Algorithm
\item{Lab: Estimating rates in simple SIS model}
\begin{itemize}
\item Goals: Data simulation and parameter estimation from complete data in a simple SIS model.
\item Example: simulate one long chain in one person
%\item Data augmentation through reversible jump MCMC 
\item Estimating rates from complete data
%\item Estimating rates from panel data
\item Diagnostics
\end{itemize}
\end{itemize}


\end{document}




\bibliography{../BIB/jul87,../BIB/oct88,../BIB/mar88,../BIB/inf90,../BIB/us,../BIB/flu,../BIB/surv}


\end{document}

\vspace{.2cm}

\noindent
{\bf Learning Objectives:} At the end of this course, the participant should be able to 
\begin{enumerate}
\item Draft
\end{enumerate} 


\vspace{.2cm}
\noindent
{\bf Textbooks and readings:}
 \begin{itemize}
 \item A
\item B
\end{itemize}




